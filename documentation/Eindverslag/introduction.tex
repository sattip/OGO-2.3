Informatici in de hedendaagse wereld komen vaak in aanraking met het ontwerpen van complexe programma's. Complexe programma's hebben vaak ook een netwerk aspect en een grafische aspect. Zelfs bij technische programma's, die vaak minder grafisch intensief zijn, zijn deze aspecten al vaak belangrijk. Dit geeft duidelijk de noodzaak aan dat elke informaticus kennis heeft van computergrafiek, computernetwerken en het ontwerpen van complexe programma's. Al deze aspecten worden verenigd in dit project: \textsc{ogo} 2.3 - Nethunt.

Voor dit project werd ons gevraagd om een interactief, gedistribueerd 3D-spel te ontwikkelen. Dit spel moest over een \textsc{lan} kunnen worden gespeeld. Wij hebben ervoor gekozen om spelers ook de mogelijkheid te geven om over het internet te spelen. Een eis bij de implementatie was een vorm van symmetrie tussen de verschillende spelers. Hier bedoelen we mee dat elke speler lokaal dezelfde spelsituatie heeft. Bovendien staan de verschillende spelers op gelijke voet gedurende het spel. Er mag dus geen server aanwezig zijn tijdens het spel. Een andere eis was dat er een vorm van voedsel aanwezig moet zijn. We willen niet dat verschillende spelers hetzelfde voedsel kunnen oppakken, dus er is een vorm van mutual exclusion vereist.

Dit document bevat de documentatie voor dit project. We zullen beginnen door een strakke planning te geven in de vorm van een werkplan. Daarna gaan we verder met een specificatie van het spel. Hierbij beschrijven we ook de alternatieven en de motivering voor onze keuze. We zijn dan klaar om een ontwerp van het spel te geven, inclusief het communicatieprotocol. De impliciete aannamen in het communicatieprotocol zullen we proberen te identificeren. We sluiten af met een validatie van deze aannamen en een motivering voor de implementatie, gevolgd door een conclusie en evaluatie. In de appendices is nog verdere documentatie opgenomen. Voor het lezen van het document kan het nuttig zijn om appendix \ref{app:begrippen} te bestuderen, waar een volledige begrippenlijst is opgenomen.

Bij de implementatie hebben we gebruik gemaakt van \emph{Doxygen}. Dit programma genereert automatisch documentatie op basis van commentaar in de code en de algemene structuur. Deze documentatie is apart toegevoegd op de \textsc{cd-rom} als Doxygen.pdf. 